% Chapter Template

\chapter{Conclusión} % Main chapter title

\label{Chapter6} % Change X to a consecutive number; for referencing this chapter elsewhere, use \ref{ChapterX}
En este proyecto, se ha realizado una vasta investigación para adquirir los conocimientos relacionados particularmente con el esquema denominado \textit{Software Defined Networking} (SDN). Realizando una comparación entre las principales implementaciones abiertas de un controlador SDN que existen hoy en día. Luego, se propuso utilizar como caso de prueba una aplicación de distribución de contenido, implementada exclusivamente utilizando las facilidades que brinda SDN. Para materializar esto, se construyó un ambiente de trabajo conteniendo el emulador de red Mininet y el controlador SDN ONOS.

A continuación, se listan las principales reflexiones finales obtenidas tras el desarrollo de este proyecto.

\begin{itemize}

    % investigación sobre SDN, creamos un documento
    \item Al finalizar este trabajo, se adquirió un conocimiento acabado del nuevo esquema de red que plantea SDN. Se ha generado un documento donde se recopilan todas las características, el principio de funcionamiento y beneficios de este innovador y prometedor esquema de red. A su vez, este escrito contiene un análisis de las dos principales implementaciones abiertas de un controlador SDN, \textit{OpenDaylight} y \textit{Open Networking Operating System} (ONOS), mostrando una comparación entre sus distintos enfoques de diseño y como implementan las diversas cualidades de un controlador SDN.
    
    % uso de mininet
    \item En el proyecto integrador se han emulado redes reales empleando la herramienta Mininet. Aquí, se ofrecen los aspectos mínimos necesarios para utilizar el emulador tanto en el escenario que este proyecto desarrolla, como en otros ámbitos relacionados con redes de computadoras.

    % estudio de multicast para la aplicación
    \item Para construir el vehículo de prueba, una aplicación de distribución de contenido, se investigó sobre el comportamiento del tráfico \textit{multicast} y sus principales protocolos involucrados, IGMP y PIM. Así, logramos exitosamente cumplir con el objetivo  de crear una aplicación capaz de entender y administrar el tráfico \textit{multicast} dentro de una red, utilizando las facilidades que brinda SDN.
    
    % uso del controlador ONOS
    \item La aplicación mencionada en el ítem anterior fue integrada en un controlador ONOS. Este hecho indujo a familiarizarse con los detalles específicos de la implementación de ONOS, los componentes que contiene, las interfaces que provee, el flujo de trabajo para insertar una aplicación personalizada, etcétera. El conocimiento del funcionamiento de este controlador fue tal que, como detallamos en la sección \ref{problemasylimi}, se logró encontrar algunos defectos precisos en él.

    % nodo físico
    \item Con la finalidad de darle un mayor sustento al desarrollo de este proyecto, se logró integrar un nodo físico a la red emulada con Mininet. Esto habla de la versatilidad de la herramienta Mininet para trabajar en conjunto al hardware de red y como para el controlador SDN le resulta transparente la implementación de los nodos de red.

\end{itemize}


\section{Problemas y limitaciones detectadas} \label{problemasylimi}

A continuación, presentaremos algunos de los problemas y limitaciones con las que nos hemos topado durante el desarrollo del proyecto,

\begin{itemize}
    % recursos de hardware. interfaces de red física. memoria ram. procesador. 
    \item Una limitación fueron los recursos de hardware necesarios para implementar el entorno de trabajo. Fue necesario que la máquina donde corre ONOS, Mininet y el servidor web, contenga un mínimo de \textit{8GB} de RAM, un procesador Intel Core i5 y 4 interfaces de red físicas, para así poder soportar una topología más compleja con gran intercambio de mensajes.  
    
    A su vez, es necesario agregarle un mínimo de dos interfaces físicas de red a la \textit{Raspberry Pi} para lograr la integración con la red emulada. Sin embargo, por cuestiones de funcionalidad y diseño, en el trabajo se optó por añadirle cuatro interfaces de red extra al embebido.
    
    % sensible a las versiones que uses del controlador.
    \item Al ser una tecnología nueva, el controlador ONOS está recibiendo modificaciones constantemente. Desde su primera entrega en diciembre de 2014, existen trece versiones distintas del controlador. Esto presenta un problema, a la hora de desarrollar una aplicación interna al controlador, uno debe prestar atención a la versión del mismo que está utilizando ya que, por el momento, ejecutar una misma aplicación en otra versión del controlador requiere modificaciones no despreciables en la aplicación.     
    
    % gestión de los \textit{intents} por el controlador.
    \item Sin duda uno de los problemas que más ralentizó el avance del proyecto fue que en el transcurso del desarrollo de la aplicación, se detectó una anomalía en la gestión de \textit{intents} por parte del controlador ONOS. Cómo se indicó en la sección \ref{sec:creacion_rutas_multicast}, es importante respetar el orden de los eventos de eliminación e instalación de \textit{intents}, pero observamos que al aumentar la frecuencia en que ocurrían estos eventos esporádicamente se perdía el sincronismo. La razón es que el mecanismo de gestión de \textit{intents} de ONOS genera un hilo para atender cada evento y al incrementar considerablemente la ocurrencia de estos eventos ocasionalmente se perdía el orden de los eventos y provocaba comportamientos indeseados en la red. Para solucionar este inconveniente, se utiliza una cola en la aplicación para corroborar que el orden de eliminación e instalación de \textit{intents} sea respetado.    

    % la GUI del controlador necesitas recargarla a veces.
    \item Otro problema menor encontrado fue que la interfaz gráfica que brinda el controlador ONOS, donde se puede ver la topología controlada y el tráfico en ella, ante sucesivos cambios abruptos en la topología o en el tráfico queda desactualizada y es necesario recargarla. Sin embargo, al ser una interfaz web, no representa gran inconveniente.
    
\end{itemize}

\section{Continuidad del trabajo} % \section{Propuestas de mejoras}

En el transcurso del trabajo han surgido propuestas para mejorar y continuar en el ámbito de este proyecto integrador. Algunas de ellas son,

\begin{itemize}

	% funcionalidad de la app para más de una red (PIM) % ASM % seguir estándar para las apps % comparación con red totalmente física.
    \item \textbf{Mejoras de la aplicación.} Como primera mejora de la aplicación de distribución de contenido se propone ampliar su funcionalidad para que soporte la administración de tráfico \textit{multicast} en más de una subred, para ello será necesario implementar el protocolo PIM en la aplicación. También, una posible mejora sería implementar una política o criterio para la selección de fuentes en el caso de suscripciones ASM, ya que el oyente no indica el transmisor, es la aplicación quién decide esto. En la versión actual de la aplicación, la fuente elegida es la primera que se registró para transmitir a esa dirección \textit{multicast}, asimismo si esta fuente falla, la recepción se interrumpe a pesar de que pueda existir otra fuente transmitiendo a esa dirección.

    Observando el resto de las aplicaciones que provee ONOS, detectamos que todas poseen una estructura de directorios y archivos común. En el desarrollo de la aplicación no se siguió esta estructura, lo cual es una posible mejora para la implementación. Por último, para mejorar el trabajo desarrollado en este proyecto, se propone comprobar el funcionamiento de la aplicación de distribución de contenido en una red con todos sus nodos físicos.
    
    % CORD
    \item \textbf{Open CORD.} \textit{Central Office Re-Architected as a Datacenter} (CORD) es un proyecto abierto de \textit{Open Networking Foundation} (ONF) que propone aprovechar los beneficios de combinar de SDN y NFV, introducidos en la sección \ref{sect:nfv}. Para lograr esto, se utilizan varias herramientas como el controlador SDN ONOS y Openstack o Docker como administradores de NFV, entre otros. De este modo CORD, aprovechando las ventajas del modelado de sistemas, pretende estandarizar los dispositivos físicos y brindar completa programabilidad a las arquitecturas de \textit{central office} de las empresas de telecomunicaciones \parencite{cord}.

    Este nuevo campo de aplicación de SDN resulta en especial atractivo ya que está siendo ampliamente aceptado entre los grandes operadores y prestadores de servicios. Entonces, es necesario primero comprender la arquitectura que propone CORD para luego poder desarrollar aplicaciones sobre esta plataforma, de manera análoga al estudio que se realizó durante este trabajo sobre ONOS.
    
    % Redes Ópticas
    \item \textbf{Redes ópticas.} Entendiendo que las redes ópticas representan un elemento clave en las comunicaciones modernas, SDN busca extender sus funcionalidades para cubrir este campo. Particularmente, en la documentación de ONOS \parencite{onosPacketOptical}, encontramos disponible una sección donde describen los proyectos y sus avances para llevar los beneficios de SDN a la capa óptica. Sin duda, esta vía de trabajo resulta muy necesaria para continuar mejorando la arquitectura de las redes contemporáneas.

\end{itemize}

\section{Aporte personal}

A modo de cierre de este proyecto integrador de la carrera de Ingeniería en Computación, se exponen algunas opiniones y valoraciones personales que han surgido con la finalización de este trabajo.

El proyecto integrador tiene como objetivo desarrollar e integrar los conocimientos adquiridos y la formación lograda a lo largo de la carrera. Es el punto culmine de un camino de 5 años para la obtención del título de grado de Ingeniero/a en Computación.

Durante la ejecución del trabajo de fin de carrera, se presentaron diversas situaciones donde fue necesario demostrar creatividad, constancia, responsabilidad y criterio profesional para hacer frente a ellas.   

Con la ayuda de los múltiples conocimientos en redes de computadoras, programación orientada a objetos, y en ingeniería de software, incorporados durante la carrera, se logró diseñar e implementar el sistema deseado. En materia de investigación y aprendizaje, más allá de la adquisición de los conceptos del innovador esquema de redes definidas por software, se realizó un estudio de los protocolos \textit{multicast}, de los cuales se tenía un conocimiento insuficiente para llevar acabo su implementación en una aplicación SDN. También, ha sido necesario aprender a utilizar Mininet y aprovechar al máximo su potencial para emular redes reales.

Para finalizar, con perseverancia y esfuerzo, se ha cumplido con éxito un proyecto desafiante sobre tecnología de vanguardia, alcanzando así una satisfacción tanto a nivel personal como profesional. 