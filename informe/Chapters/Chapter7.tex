% Chapter Template

\chapter{Conclusión} % Main chapter title

\label{Chapter7} % Change X to a consecutive number; for referencing this chapter elsewhere, use \ref{ChapterX}
En este proyecto, se ha realizado una vasta investigación para adquirir conocimientos relacionados particularmente con la gestión de la configuración de los equipos de red y también con el esquema denominado Redes Definidas por Software. Así, se realizó una comparación entre dos implementaciones abiertas del protocolo \textit{NETCONF} y un estudio del controlador \textit{ONOS}. Luego, se propuso utilizar ambas tecnologías con el fin de poder monitorear y configurar un dispositivo óptico, precisamente un \textit{muxponder} de 40GB. Para materializar esto, se construyó un ambiente de trabajo conformado por el controlador \textit{ONOS}, \textit{switchs} virtuales, computadoras de propósito general e integrando el agente \textit{NETCONF} \textit{Yuma123} en el \textit{muxponder} de 40GB.

A continuación, se listan las principales reflexiones finales obtenidas tras el desarrollo de este proyecto.

\begin{itemize}

    % investigación sobre SDN, creamos un documento
    \item Al finalizar este trabajo, se adquirió un conocimiento acabado del nuevo esquema de red que plantea  \textit{SDN}. Se ha generado un documento donde se recopilan todas las características, el principio de funcionamiento y beneficios de este innovador y prometedor enfoque. A su vez, este escrito contiene un análisis del controlador \textit{ONOS}, mostrando su diseño y como implementa este las diversas cualidades de un controlador  \textit{SDN}.

    % uso de netconf
    \item En el proyecto integrador se ha realizado un estudio sobre la gestión de la configuración de los equipos de red. Se logró integrar y adaptar al \textit{muxponder} una implementación de código abierto del protocolo \textit{NETCONF} llamada Yuma123. 
    
    \item Con el fin de poder administrar la configuración del equipo, se realizó un módulo \textit{YANG} que describe y modela los datos del dispositivo. También fue necesario realizar una librería en \textit{C} que relacione los comportamientos típicos y las variables del equipo con el agente Yuma123.
        
    % uso del controlador ONOS
    \item Para lograr una comunicación entre los dispositivos y el controlador, se desarrolló un \textit{driver} en la interfaz \textit{southbound} de \textit{ONOS}. Además, se realizó también una aplicación que reside en la interfaz \textit{northbound} del mismo, con el fin de exponer interfaces a las aplicaciones externas para que estas puedan comunicarse con los dispositivos. 

    % nodo físico
    \item Haciendo uso de lenguajes como \textit{Python}, \textit{HTML}, \textit{JavaScript} y \textit{CSS}, se desarrolló una aplicación que presenta una interfaz gráfica al administrador. En la misma, se muestra información de los dispositivos y permite al administrador realizar cambios en la configuración. Para ello, dicha aplicación se comunica a través de la interfaz \textit{northbound} del controlador. 

\end{itemize}

 Finalmente al concluir el presente proyecto de fin de grado, el \textit{muxponder} obtuvo las siguientes ventajas:

 \begin{itemize}

    \item Poder realizar cambios en la configuración del dispositivo de forma remota, segura y orientada a la conexión a través del protocolo \textit{NETCONF}. 
    \item El equipo ahora es capaz de integrarse correctamente a un ambiente \textit{SDN}, obteniendo así las ventajas de escalabilidad, programabilidad, monitoreo y seguridad que ofrecen las redes \textit{SDN}.
    \item Por otro lado, el controlador utilizado en el proyecto permirte visualizar de forma gráfica el estado de los enlaces entre los diferentes equipos conectados, lo que permite al administrador conocer e identificar de forma sencilla el estado de los mismos.
    \item Además, el soporte de un entorno \textit{SDN} por parte del \textit{muxponder}, permite al proveedor de servicios diseñar y adaptar nuevas aplicaciones que interactúen con los dispositivos a través de la capa \textit{northbound} del controlador, evitando la necesidad de introducir mayores cambios en los equipos e impactando en la disponibilidad de los servicios que ofrece.
    \item Por último, el soporte de \textit{SDN} ofrece un valor agregado al equipo para los operadores de red, permitiendo que el mismo pueda operar junto a otros dispositivos de diferentes fabricantes. De esta manera, el problema de aprovisionamiento de servicios en la red se vuelve un problema exclusivamente de software, lo que les da libertad a los operadores al momento de implementar cambios en la red.

\end{itemize}


\section{Problemas y limitaciones} \label{problemasylimi}

A continuación, se presentan algunos de los problemas y limitaciones del proyecto:

\begin{itemize}
    
    % sensible a las versiones que uses del controlador.
    \item Al ser una tecnología nueva, el controlador \textit{ONOS} recibe modificaciones constantemente. Desde su primera entrega en diciembre de 2014, existen diecisiete versiones distintas del controlador. Así, a la hora de desarrollar una aplicación interna al controlador, esto presenta un problema ya uno debe prestar atención a la versión del mismo que está utilizando porque por el momento, ejecutar una misma aplicación en otra versión del controlador requiere modificaciones no despreciables.     
    
    % limitaciones del dispositivi.
    \item Sin duda uno de los problemas que más ralentizó el avance del proyecto fueron las limitaciones en cuanto a recursos disponibles del dispositivo utilizado, el \textit{muxponder} de 40GB. Se tuvo dificultades a la hora de integrar algún agente del protocolo \textit{NETCONF} debido a las capacidades del equipo y las diferentes librerías necesarias por las implementaciones. Precisamente, se tuvieron problemas con la utilización de la memoria principal y secundaria del \textit{muxponder}, como así también la arquitectura que presenta el procesador \textit{NIOS II}.

    % la GUI del controlador necesitas recargarla a veces.
    \item Otro problema menor encontrado fue la interfaz gráfica que brinda el controlador \textit{ONOS}, donde se puede ver que la topología controlada ante sucesivos cambios abruptos en ella queda desactualizada y es necesario recargarla. Sin embargo, al ser una interfaz web, no representa gran inconveniente.
    
    % problema inicio sesion controlador.
    \item Por último, se detectó un problema a la hora de establecer el inicio de sesión y el intercambio de capacidades entre el controlador y el agente \textit{NETCONF} del dispositivo. Sucede que el inicio de sesión entre ambos demora unos segundos debido a las capacidades del \textit{muxponder}, y durante ese momento el controlador no bloquea otras operaciones \textit{NETCONF} con el dispositivo. Esto provocaba que el servidor del dispositivo cierre inmediatamente la sesión \textit{SSH} con el controlador, por lo que para solucionar el inconveniente se tuvo que realizar una espera en la funcion \textit{DeviceDescriptionDiscovery} hasta que el intercambio finalice correctamente antes de llamar cualquier otra operación \textit{NETCONF}.
    
\end{itemize}

\section{Continuidad del trabajo} % \section{Propuestas de mejoras}

En el transcurso del trabajo han surgido propuestas para mejorar y continuar en el ámbito de este proyecto integrador. Algunas de ellas son:

\begin{itemize}

    
    \item \textbf{Optimizaciones en el \textit{driver}.} Observando el resto de las aplicaciones y \textit{drivers} que provee ONOS, se detectó que todas poseen una estructura de directorios y archivos común. En el desarrollo del \textit{driver} no se siguió estrictamente esta estructura, lo cual es una posible mejora para la implementación.

    \item \textbf{Adaptación del módulo \textit{YANG} a \textit{OpenConfig}.} \textit{OpenConfig} propone un modelo que busca ser estandarizado por la \textit{IETF} para homogeneizar los datos de configuración y estado de los dispositivos de red, entre ellos dispositivos ópticos como el \textit{muxponder}. De esta forma, adaptar el esquema que propone \textit{OpenConfig} al módulo \textit{YANG} desarrollado supondría una ventaja a los proveedores de servicios que requieren una comunicación con diversos dispositivos de diferentes fabricantes.
    
    \item \textbf{Auto descubrimiento de los vecinos.} El procedimiento que el controlador lleva para formar los enlaces de los dispositivos dependen de la información y la configuración de los mismos (es decir, el equipo debe contar con información de número de serie del vecino, puerto vecino y puerto local para formar un enlace). Una posible mejora podría ser mejorar este proceso para hacerlo independiente de la configuración y por lo tanto independiente del administrador, haciendo que los dispositivos auto descubran sus vecinos. 

\end{itemize}

\section{Aporte personal}

A modo de cierre de este proyecto integrador de la carrera de Ingeniería en Computación, se exponen algunas opiniones y valoraciones personales que han surgido con la finalización de este trabajo.

El proyecto integrador tiene como objetivo desarrollar e integrar los conocimientos adquiridos y la formación lograda a lo largo de la carrera. Es el punto culmine de un camino de 5 años para la obtención del título de grado de Ingeniero en Computación.

Durante la ejecución del trabajo de fin de carrera, se presentaron diversas situaciones donde fue necesario demostrar creatividad, constancia, responsabilidad y criterio profesional para hacer frente a ellas.   

Con la ayuda de los múltiples conocimientos en redes de computadoras, programación orientada a objetos e ingeniería de software incorporados durante la carrera, se logró diseñar e implementar el sistema deseado. En materia de investigación y aprendizaje, más allá de la adquisición de los conceptos del innovador esquema de redes definidas por software, se realizó un estudio de los protocolos de gestión de la configuración de los dispositivos de red, entre ellos \textit{NETCONF}, de los cuales se tenía un conocimiento insuficiente para llevar acabo su implementación en un entorno \textit{SDN}.

Para finalizar, con perseverancia y esfuerzo, se ha cumplido con éxito un proyecto desafiante sobre tecnología de vanguardia, alcanzando así una satisfacción tanto a nivel personal como profesional. 