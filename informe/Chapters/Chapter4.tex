% Chapter Template

\chapter{Entorno de trabajo} % Main chapter title
% cSpell:words resizebox onosapi usecase Mcast enrutamiento parencite includegraphics pcktprocactiv lstlisting iperf

\label{Chapter4} % Change X to a consecutive number; for referencing this chapter elsewhere, use 
% \ref{ChapterX} 
Teniendo en cuenta las tecnologías analizadas anteriormente, claro está que para llevar a cabo el desarrollo de las aplicaciones es necesario implementar un sistema en donde se puedan realizar las pruebas necesarias y evaluar de este modo el desempeño de las mismas. Primero se determinarán los requerimientos de este sistema, la estructura y su comportamiento; una vez caracterizado el sistema, se implementa y prueba su correcto funcionamiento. 

Después de comprobar el correcto comportamiento del sistema, se analizan las herramientas a utilizar para el desarrollo de las aplicaciones dentro de este ambiente de prueba. 

%----------------------------------------------------------------------------------------
%	SECTION 1
%----------------------------------------------------------------------------------------

\section{Infraestructura empleada}
Se detallarán a continuación las características del sistema a implementar, descriptas en el lenguaje de especificación de sistemas SysML \parencite{sysml}. Se optó por este lenguaje de modelado ya que brinda una extensión a UML permitiendo combinar elementos del mundo físico (\textit{hardware}) con elementos del mundo lógico (\textit{software}). 

%-----------------------------------
%	SUBSECTION 1
%-----------------------------------

\subsection{Requerimientos del sistema}

% \subsubsection*{Casos de uso}
