% Chapter Template

\chapter{Metodologías y Herramientas de Trabajo} % Main chapter title
% cSpell:words resizebox \textit{onosapi} usecase Mcast enrutamiento parencite includegraphics pcktprocactiv lstlisting iperf

\label{Chapter4} % Change X to a consecutive number; for referencing this chapter elsewhere, use 
% \ref{ChapterX} 
En la actualidad, existen diversas metodologías para el desarrollo del software separadas en dos grandes grupos: metodologías tradicionales y metodologías ágiles. 

Será importante definir y adoptar alguna de ellas con el fin de establecer convenciones para el desarrollo del proyecto integrador. Se aborda así en la primer sección de este capítulo, la metodología de trabajo utilizada en las diferentes etapas.

Luego, se definen las herramientas de trabajo empleadas durante el desarrollo del proyecto y finalmente se describe la planificación de riesgos.

%----------------------------------------------------------------------------------------
%	SECTION 1
%----------------------------------------------------------------------------------------

\section{Metodología de desarrollo de software}
Para la gestión del desarrollo del software de este proyecto, se utilizó la metodología ágil iterativa e incremental. Dicha metodología consiste en dividir el proyecto en iteraciones bien definidas. En cada una de estas iteraciones se agregan nuevas funcionalidades y también se pueden introducir mejoras sobre las iteraciones anteriores. Los factores determinantes para la decisión de esta metodología se listan a continuación:

\begin{itemize}
	\item Existe la necesidad de tener una metodología en la cual se realicen reportes al director y co-director del proyecto, quienes analizarán los avances.
	\item Al dividir el proyecto en etapas, permite centrar la atención y organización a una pequeña parte de la misma.
	\item La necesidad de tener una metodología que sea flexible ante requerimientos cambiantes.
\end{itemize}
%-----------------------------------
%	SUBSECTION 1
%-----------------------------------

\subsection{Organización de las iteraciones}
Una vez definida la metodología para la gestión del proyecto, se definieron en conjunto con el director y co-director las iteraciones que conformarán el proyecto. Las mismas se presentan a continuación.
\begin{itemize}
	\item \textbf{Iteración 1: Familiarización con el \textit{muxponder}.} El objetivo de esta iteración es entender cómo funciona el equipo, sus capacidades, aplicaciones y utilidades.
	\item \textbf{Iteración 2: Investigación del protocolo \textit{NETCONF}.} Esta iteración tiene como objetivo realizar un estudio sobre el protocolo \textit{NETCONF}, las distintas implementaciones disponibles y cómo utilizar las mismas.
	\item \textbf{Iteración 3: Adaptación del agente \textit{NETCONF} al \textit{muxponder}.} Esta etapa del proyecto consiste en adaptar algún agente al equipo, realizando pruebas sobre las funcionalidades, capacidades y limitaciones del mismo.
	\item \textbf{Iteración 4: Realización de la librería en el \textit{muxponder}.} Independientemente del agente utilizado en el equipo, se deberá realizar una aplicación que relacione el agente \textit{NETCONF} con la instrumentación del \textit{muxponder}.
	\item \textbf{Iteración 5: Desarrollo del \textit{driver} en el controlador.} El objetivo de esta iteración será la de realizar un \textit{driver} en el controlador para que el mismo pueda comunicarse con los diferentes \textit{muxponders} presentes en la topología.
	\item \textbf{Iteración 6: Realización de la aplicación API REST en el controlador.} En esta etapa, se desarrolla la aplicación \textit{REST} que expone las distintas funcionalidades implementadas en el \textit{driver} del controlador, para que aplicaciones externas puedan hacer uso del mismo.
	\item \textbf{Iteración 7: Desarrollo de la interfaz gráfica.} Finalmente, en esta iteración se desarrolla la interfaz gráfica, la cual a través de la API REST desarrollada, se comunica con los diferentes equipos haciendo uso del controlador y las bondades de \textit{SDN}.
\end{itemize}

\section{Lenguajes de desarrollo}
En el presente trabajo de fin de grado, se utilizaron diversos lenguajes de programación. Se presenta a continuación una lista de los lenguajes empleados como así también la etapa en la cual fueron aplicados. 

\begin{itemize}
	\item \textbf{C}. El agente Yuma123 y la librería en C que comunica la instrumentación del dispositivo con el agente, se encuentran desarrollados en dicho lenguaje.  
	\item \textbf{YANG}. La librería mencionada en el punto anterior, hará uso del lenguaje de modelado conocido como YANG para describir los objetos presentes en \textit{NETCONF}. 
	\item \textbf{Java}. Tanto el controlador \textit{ONOS}, como el \textit{driver} y la \textit{API REST} que se desarrolló, se encuentran escritos en dicho lenguaje. 
	\item \textbf{Python}. Durante el diseño de la interfaz gráfica, se optó por \textit{Flask} como \textit{web server}, el cual consiste en un \textit{framework} minimalista escrito en \textit{Python} que permite crear aplicaciones \textit{web} de forma rápida y sencilla. 
	\item \textbf{HTML}. La interfaz gráfica además del \textit{web server} escrito en \textit{Python}, utiliza \textit{HTML} como lenguaje de marcado.  
	\item \textbf{LaTeX}. Utilizado para la documentación del proyecto y la generación del informe del trabajo de fin de grado. 
\end{itemize}

\section{Herramientas de desarrollo}
Las herramientas que conformarán el entorno de desarrollo se listan a continuación.

\begin{itemize}
	\item \textbf{IntelliJ IDEA} como entorno de desarrollo utilizado para proyectos \textit{Java} y \textit{Python}.  
	\item \textbf{Maven} para la gestión de proyectos \textit{Java}, empleado especificamente en la construcción del controlador \textit{ONOS}.
	\item \textbf{Visual Studio Code} como editor de código fuente, utilizado en multiples etapas del proyecto.
\end{itemize}

\section{Control de versiones}
Se optó por utilizar el sistema de control de versiones conocido como \textit{Git} \parencite{gitref} alojado en un servidor de \textit{GitHub} \parencite{githubref}. El hecho de utilizar un sistema de control de versiones supone las siguientes ventajas:

\begin{itemize}
	\item Mantiene un historial de versiones del software. De esta forma, cada cambio se encuentra versionado permitiendo que en cualquier momento se pueda regresar el código a estados anteriores o comparar los cambios entre las distintas versiones del mismo.   
	\item Acceso remoto al proyecto. Al utilizar un servidor de \textit{Git} público, el proyecto puede ser accedido de forma remota desde cualquier equipo para continuar con el desarrollo del mismo en cualquier momento.
	\item Utilización de un esquema de ramas. Permite trabajar de manera paralela sobre diferentes partes del proyecto sin interferencias, permitiendo integrar nuevas funcionalidades al software de manera segura.
\end{itemize}

\section{Planificación de riesgos}
Todo desarrollo de software implica una serie de riesgos e incertidumbres que se deben planificar para definir una forma estructurada de actuar ante ellas, con el fin de cumplir con los tiempos establecidos.
  
El objetivo es identificar y analizar los riesgos que se puedan presentar para establecer estrategias de control y resolución que permitan ejercer una correcta supervisión de los mismos.

Por lo tanto, el proceso de gestión de riesgos está compuesto por los siguientes \textit{items}:

\begin{itemize}
	\item Identificación del riesgo.
  \item Evaluación de su probabilidad de aparición.
  \item Estimación del impacto.
  \item Establecer un plan de contingencia en caso de que ocurra.
\end{itemize}

\subsection{Criterio}
A continuación se explica el criterio utilizado para el análisis de riesgos. Los riesgos deben ser clasificados según su probabilidad de ocurrencia y los efectos que produzcan en función a los retrasos en los plazos del proyecto. 

De esta forma, se tiene el cuadro \ref{tab:probabilidad_riesgo} el cual describe el criterio utilizado para la probabilidad de ocurrencia de los riesgos.

\begin{table}[H]
  \centering
  \begin{tabular}{|c|c|}
  \hline
                    & \textbf{Probabilidad} \\ \hline
  \textbf{Muy baja} & \textless 10\%        \\ \hline
  \textbf{Baja}     & 10\% - 30\%           \\ \hline
  \textbf{Moderada} & 30\% - 70\%           \\ \hline
  \textbf{Alta}     & 70\% - 90\%           \\ \hline
  \textbf{Muy alta} & \textgreater 90\%     \\ \hline
  \end{tabular}
  \caption{Probabilidad de ocurrencia del riesgo.}
  \label{tab:probabilidad_riesgo}
  \end{table}

Mientras que el cuadro \ref{tab:efectos_riesgo} describe el criterio que se utilizó para definir el impacto de los efectos del riesgo. 

\begin{table}[H]
  \centering
  \begin{tabular}{|c|c|}
  \hline
                        & \textbf{Retraso}                        \\ \hline
  \textbf{Insignificante} & Menos de 8 horas de trabajo.            \\ \hline
  \textbf{Moderado}     & Entre 8 y 48 horas de trabajo.          \\ \hline
  \textbf{Mayor}        & Mayores a 48 horas de trabajo.          \\ \hline
  \textbf{Catastrófico}      & Puede soponer el abandono del proyecto. \\ \hline
  \end{tabular}
  \caption{Impacto de los efectos del riesgo.}
  \label{tab:efectos_riesgo}
  \end{table}

Al combinar la probabilidad de aparición y los efectos del riesgo, se obtiene un nivel de importancia del riesgo. Esto último se puede ver de forma gráfica en la figura \ref{tab:niveles_riesgo}. 

\begin{table}[H]
  \centering
  \begin{tabular}{|c|l|c|l|l|}
  \hline
  \textbf{Probabilidad/Efecto} & \textbf{Insignificante}                           & \textbf{Moderado}        & \textbf{Mayor}           & \textbf{Catastrófico}         \\ \hline
  \textbf{Muy baja}            & \cellcolor[HTML]{9AFF99}{\color[HTML]{FFFFFF} }\textbf{Importancia 1} & \cellcolor[HTML]{9AFF99} \textbf{Importancia 1} & \cellcolor[HTML]{9AFF99} \textbf{Importancia 2} & \cellcolor[HTML]{FFCC67} \textbf{Importancia 3} \\ \hline
  \textbf{Baja}                & \cellcolor[HTML]{9AFF99} \textbf{Importancia 1}                        & \cellcolor[HTML]{9AFF99} \textbf{Importancia 2} & \cellcolor[HTML]{FFCC67} \textbf{Importancia 3} & \cellcolor[HTML]{FFCC67} \textbf{Importancia 4} \\ \hline
  \textbf{Moderada}            & \cellcolor[HTML]{9AFF99}  \textbf{Importancia 2}  & \cellcolor[HTML]{FFCC67} \textbf{Importancia 3} & \cellcolor[HTML]{FFCC67} \textbf{Importancia 4} & \cellcolor[HTML]{CB0000} \textbf{Importancia 5} \\ \hline
  \textbf{Alta}                & \cellcolor[HTML]{FFCC67} \textbf{Importancia 3} & \cellcolor[HTML]{FFCC67} \textbf{Importancia 4} & \cellcolor[HTML]{CB0000} \textbf{Importancia 5} & \cellcolor[HTML]{CB0000} \textbf{Importancia 6} \\ \hline
  \textbf{Muy alta}            & \cellcolor[HTML]{FFCC67} \textbf{Importancia 4} & \cellcolor[HTML]{CB0000} \textbf{Importancia 5} & \cellcolor[HTML]{CB0000} \textbf{Importancia 6} & \cellcolor[HTML]{CB0000} \textbf{Importancia 6} \\ \hline
  \end{tabular}
  \caption{Impacto de los efectos del riesgo.}
  \label{tab:niveles_riesgo}
  \end{table}

\subsection{Identificación y análisis de los riesgos}

Según \parencite{ingsoftian} se puede clasificar a los riesgos en los siguientes tipos:

\begin{itemize}
  \item \textbf{Riesgos de personal.} Aquellos riesgos relacionados con el equipo que conforma el proyecto.
	\item \textbf{Riesgos de tecnología.} Son aquellos riesgos los cuáles están asociados a las herramientas de hardware utilizadas.
  \item \textbf{Riesgos organizacionales.} Todo riesgo relacionado al entorno organizacional del proyecto.
  \item \textbf{Riesgos de requerimientos.} Son todos los riesgos asociados con los cambios de requerimientos del cliente.
  \item \textbf{Riesgos de estimación.} Riesgos vinculados a estimaciones administrativas del proyecto.
  \item \textbf{Riesgos de herramientas.} Aquellos riesgos vinculados a las herramientas de software.
\end{itemize}

A continuación, se muestra en la tabla \ref{tab:riesgos_identificados} los distintos riesgos identificados que podrían presentarse durante el desarrollo del proyecto.

\begin{table}[H]
  \rowcolors{2}{white!10!gray!80}{white!80!gray!40}
  \centering
  \begin{tabular}{ |m{2.5cm}|m{7.5cm}|m{2.5cm}|  }
  
  \hline
  \centering
  \textbf{Identificador} & \textbf{Descripción} & \textbf{Clasificación} \\
  \hline
  \centering
  Riesgo 1 & Se subestima el tamaño del proyecto. & Estimación  \\
  \hline
  \centering
  Riesgo 2 & Cambios en los requerimientos. & Requerimiento  \\
  \hline
  \centering
  Riesgo 3 & La versión del controlador \textit{SDN} no es compatible con alguna herramienta utilizada en el proyecto. & Herramientas \\
  \hline
  \centering
  Riesgo 4 & La versión del agente \textit{NETCONF} no es compatible con las herramientas utilizadas en el proyecto. & Herramientas  \\
  \hline

  \centering
  Riesgo 5 & Alguno de los módulos \textit{XFP} que comunican los clientes con los \textit{muxponder} se dañan.  & Tecnológico  \\
  \hline

  \centering
  Riesgo 6 & Las librerías instaladas en el equipo no son compatibles con el agente. & Herramientas  \\
  \hline

  \centering
  Riesgo 7 & Las fibras del equipo se dañan. & Tecnológico  \\
  \hline

  \centering
  Riesgo 8 & Se daña fisicamente algún chip del equipo. & Tecnológico  \\
  \hline
  
  \end{tabular}
  \caption{Identificación de los riesgos}
  \label{tab:riesgos_identificados}
  \end{table}

  Luego, una vez identificados los riesgos se procede a analizarlos, esto es, estimar tanto la probabilidad de ocurrencia del riesgo como el efecto del mismo, para definir finalmente un grado de importancia. La tabla \ref{tab:analis_identificados} muestra el resultado de combinar ambas estimaciones haciendo uso del cuadro presentado en \ref{tab:niveles_riesgo}.

  \begin{table}[H]
    \centering
    \begin{tabular}{|c|c|c|c|}
    \hline
    \textbf{Identificador} & \textbf{Probabilidad de ocurrencia} & \textbf{Efecto} & \textbf{Grado de importancia}                                       \\ \hline
    \textbf{Riesgo 1}      & Alta                                & Mayor           & \cellcolor[HTML]{CB0000}\textbf{Importancia 5}                        \\ \hline
    \textbf{Riesgo 2}      & Moderada                            & Moderada        & \cellcolor[HTML]{F8A102}\textbf{Importancia 3}                        \\ \hline
    \textbf{Riesgo 3}      & Moderada                            & Moderada        & \cellcolor[HTML]{F8A102}\textbf{Importancia 3}                        \\ \hline
    \textbf{Riesgo 4}      & Baja                                & Moderada        & \cellcolor[HTML]{9AFF99}{\color[HTML]{000000} \textbf{Importancia 2}} \\ \hline
    \textbf{Riesgo 5}      & Baja                                & Mayor           & \cellcolor[HTML]{F8A102}\textbf{Importancia 3}                        \\ \hline
    \textbf{Riesgo 6}      & Moderada                            & Moderada        & \cellcolor[HTML]{F8A102}\textbf{Importancia 3}                        \\ \hline
    \textbf{Riesgo 7}      & Baja                                & Mayor           & \cellcolor[HTML]{F8A102}\textbf{Importancia 3}                        \\ \hline
    \textbf{Riesgo 8}      & Muy baja                            & Mayor         & \cellcolor[HTML]{9AFF99}{\color[HTML]{000000} \textbf{Importancia 2}} \\ \hline
    \end{tabular}
    \caption{Análisis de los riesgos}
    \label{tab:analis_identificados}
    \end{table}

\subsection{Estrategias de gestión de los riesgos}

Por último, se procede a realizar un plan que contenga posibles soluciones a los riesgos analizados en las secciones anteriores. De esta forma resulta el cuadro \ref{tab:riesgos_solucion}.

\begin{table}[H]
  \rowcolors{2}{white!10!gray!80}{white!80!gray!40}
  \centering
  \begin{tabular}{ |m{2.2cm}|m{5.5cm}|m{5.5cm}|  }

\hline
\centering
\textbf{Identificador} & \textbf{Consecuencia} & \textbf{Estrategia de mitigación} \\
\hline
\centering
Riesgo 1 & Se extienden los tiempos de desarrollo. & Estimar los tiempos de desarrollo para el caso más desfavorable.  \\
\hline
\centering
Riesgo 2 & Desperdicio de tiempos de desarrollo y extensión de fecha de entrega del proyecto. & Control y revisión de requerimiento en cada iteración y en etapas tempranas de desarrollo.  \\
\hline
\centering
Riesgo 3 & Se presenta un problema para cumplir con los requerimientos funcionales del proyecto. & Mantener actualizadas las versiones de las herramientas y del controlador, utilizando siempre versiones estables. \\
\hline
\centering
Riesgo 4 & Se presenta un problema para cumplir con los requerimientos funcionales del proyecto. & Mantener actualizadas las versiones de las herramientas y del agente, utilizando siempre versiones estables.  \\
\hline

\centering
Riesgo 5 & Se extienden los tiempos de desarrollo, ya que no se podran realizar pruebas de conectividad entre clientes.  & En caso de daño, reemplazar el módulo \textit{XFP} adquiriendo uno nuevo.  \\
\hline

\centering
Riesgo 6 & Se extienden los tiempos de desarrollo del proyecto, ya que el agente responderá de forma incorrecta o impredecible. & Actualizar las librerías utilizadas por el agente en el dispositivo.  \\
\hline

\centering
Riesgo 7 & Se extienden los tiempos de desarrollo, ya que no se podran realizar pruebas de conectividad entre clientes. & En caso de daño, reemplazar la fibra defectuosa por una nueva.  \\
\hline

\centering
Riesgo 8 & Impide la utilización parcial o completa del equipo. & Mantener apagados los equipos en caso de que no se esté usando. Ante cualquier duda sobre la utilización del mismo, buscar soporte técnico.  \\
\hline

\end{tabular}
\caption{Estrategias de solución para los riesgos}
\label{tab:riesgos_solucion}
\end{table}
