% Chapter Template

\chapter{Entorno de trabajo} % Main chapter title
% cSpell:words resizebox onosapi usecase Mcast enrutamiento parencite includegraphics pcktprocactiv lstlisting iperf

\label{Chapter4} % Change X to a consecutive number; for referencing this chapter elsewhere, use 
% \ref{ChapterX} 
Teniendo en cuenta las tecnologías analizadas anteriormente, claro está que para llevar a cabo el desarrollo de las aplicaciones es necesario implementar un sistema en donde se puedan realizar las pruebas necesarias y evaluar de este modo el desempeño de las mismas. Primero se determinarán los requerimientos de este sistema, la estructura y su comportamiento; una vez caracterizado el sistema, se implementa y prueba su correcto funcionamiento. 

Después de comprobar el correcto comportamiento del sistema, se analizan las herramientas a utilizar para el desarrollo de las aplicaciones dentro de este ambiente de prueba. 

%----------------------------------------------------------------------------------------
%	SECTION 1
%----------------------------------------------------------------------------------------

\section{Infraestructura empleada}
Se detallarán a continuación las características del sistema a implementar, descriptas en el lenguaje de especificación de sistemas SysML \parencite{sysml}. Se optó por este lenguaje de modelado ya que brinda una extensión a UML permitiendo combinar elementos del mundo físico (\textit{hardware}) con elementos del mundo lógico (\textit{software}). 

%-----------------------------------
%	SUBSECTION 1
%-----------------------------------

\subsection{Requerimientos del sistema}

% \subsubsection*{Casos de uso}
Primero, antes de definir los requerimientos del sistema es necesario identificar los \textit{casos de uso}, que son la base para establecer los requerimientos funcionales. 

El diagrama de la figura \ref{fig:usecase} representa los casos de uso del sistema, como se puede ver el objeto principal del sistema es brindar al desarrollador de aplicaciones un entorno en el que sus aplicaciones puedan implementarse y probarse. En base a este diagrama se pueden definir los requerimientos funcionales a nivel de sistema. 

En la figura \ref{fig:sistreqs} se pueden ver el listado y una breve descripción de los requerimientos tanto funcionales como no funcionales. Como se mencionó anteriormente, los requerimientos funcionales se obtienen a partir de los casos de uso. Como requerimiento no funcional se puede identificar la adición de un nodo físico al sistema que deberá integrarse al emulador de red y ser programado por las aplicaciones, del mismo modo que los dispositivos emulados. Este último no afecta a la funcionalidad del sistema, pero le brinda más flexibilidad, permitiéndole funcionar fuera del entorno emulado, o conectar al emulador a un sistema externo.


\begin{figure}[th]
    \centering 
    \resizebox{0.55\textwidth}{!}{\includegraphics{Figures/UseCaseDiagram.png}}
    \caption[Diagrama de Casos de Uso SysML]{Diagrama de Casos de Uso desde el punto de vista del desarrollador de aplicaciones.}
    \label{fig:usecase}
\end{figure}

\begin{figure}[th]
    \centering 
    \resizebox{0.9\textwidth}{!}{\includegraphics{Figures/RequirementDiagram.png}}
    \caption[Diagrama de Requerimientos SysML]{Diagrama de requerimientos a nivel de sistema.}
    \label{fig:sistreqs}
\end{figure}

%-----------------------------------
%	SUBSECTION 2
%-----------------------------------

\subsection{Diagrama de bloques de los componentes del sistema}
\begin{figure}[th]
	\centering 
	\resizebox{0.9\textwidth}{!}{\includegraphics{Figures/BlockDiagram.png}}%
	\caption[Diagrama en bloques SysML]{Diagrama en bloques de los componentes del sistema.}
	\label{fig:BlockDiagram}
\end{figure}

Para lograr formar un entorno donde funcione \textit{multicast} con SDN se planteó la estructura que muestra la figura 
\ref{fig:BlockDiagram}. Los bloques \textit{Controlador ONOS, Servidor Web y Mininet} correrán en máquinas virtuales 
independientes dentro de una misma computadora. Estas 3 entidades estarán comunicadas entre sí a través de una red 
\textit{host-only}. 

Por otro lado, el bloque \textit{Nodo Físico} es implementado a través de una \textit{Raspberry Pi} la cual tendrá un sistema operativo Linux instalado con un módulo \textit{Open vSwitch} incorporado. La comunicación del nodo físico con el controlador y la red emulada será por dos interfaces distintas. Con \textit{Mininet} se establece un \textit{bridge} y con el controlador una conexión TCP.

En cuanto a las aplicaciones, \textit{IgmpSnoop} es instalada en el controlador y utilizará sus servicios para poder controlar la distribución de contenido en la red y exportará los parámetros necesarios para poder administrar la red. La aplicación web será la encargada de modificar dichos parámetros.

Finalmente, para emular la red se definirá la topología a emplear en el bloque \textit{Mininet} utilizando las librerías en \textit{Python} que el emulador brinda. A su vez, \textit{Mininet} también contiene un módulo \textit{Open vSwitch}.  

%-----------------------------------
%	SUBSECTION 3
%-----------------------------------

\subsection{Modelo de comportamiento del sistema}

Los procesos y el flujo de actividades dentro del sistema constituyen el modelo dinámico o de comportamiento del mismo. Se deben analizar para los casos 
de uso los distintos procesos que se encuentran involucrados, luego se describen los más representativos que permitan ver de una forma más clara la
 comunicación entre los distintos componentes y el flujo de la información dentro del sistema. \

En este caso se seleccionó el proceso de incursión de una configuración desde una aplicación web al controlador, la razón de modelar este proceso reside en que 
este involucra la mayor cantidad de componentes del sistema. \

En el diagrama \ref{fig:ActivDia} se refleja el flujo de actividad del proceso seleccionado. Como se puede ver, el evento inicial es la adición de una configuración desde 
la aplicación web. Esta configuración, una vez procesada se traduce en mensajes HTTP que envía hacia la \textit{northbound interface}
del controlador. Una vez recibida la consulta, se realiza su procesamiento, y en el caso de que sea necesario, se 
realizan modificaciones en las reglas de los dispositivos a través de mensajes \textit{Openflow}. \

Como dispositivo de red a configurar se puede utilizar tanto la topología emulada como el nodo físico o una combinación de ambos, involucrando de este modo 
a todos los elementos del sistema.


\begin{figure}[th]
    \centering 
    \resizebox{0.4\textwidth}{!}{\includegraphics{Figures/ActivityDiagramConfigController.png}}%
    \caption[Diagrama de Actividad SysML]{Diagrama de actividad del comportamiento esperado ante una configuración desde la aplicación web.}
    \label{fig:ActivDia}
\end{figure}

\section{Herramientas de desarrollo}

Dado que el proyecto involucra aplicaciones en distintos entornos, fue necesaria la utilización de distintas herramientas avocadas 
al desarrollo de cada una.

\subsection{Maven}
El controlador ONOS utiliza Maven \parencite{maven} para facilitar la construcción de los proyectos en Java. Maven es una herramienta de gestión de software, que facilita la compilación del código, resolución de dependencias y los pruebas. Para llevar a cabo estas tareas, utiliza un fichero POM donde están descriptas todas las características del proyecto, incluyendo las dependencias a distintos módulos y las librerías en sus respectivas versiones.
 
\subsection{Mininet Python API}
\textit{Mininet} provee una serie de clases en \textit{Python} que permiten crear una topología flexible que puede configurarse en base a los parámetros que uno les establezca \parencite{mininet}. 

\subsection{Iperf-SSM}\label{sec:iperf}
Para probar el comportamiento del sistema va a ser necesario generar tráfico \textit{multicast}, para esto se utilizará \textit{iperf}. Esta herramienta se utiliza para medir el rendimiento de la comunicación entre dos extremos. \textit{Iperf} permite crear tráfico tanto TCP como UDP y configurar varios parámetros de la conexión y algunas características de los paquetes a enviar. Por otro lado, \textit{iperf} tiene una importante limitación en cuanto a la transmisión de paquetes \textit{multicast} y reside en el soporte a mensajes SSM. Debido a esto se acude a una herramienta que provee una extensión \parencite{iperfssm}, denominada \textit{iperf-ssm}. Esta herramienta debe ser compilada e instalada en el sistema operativo.
 
A continuación, se explica brevemente como utilizar la herramienta para establecer la comunicación \textit{multicast} SSM


\begin{lstlisting}[language=bash, caption={Utilización de la herramienta iperf}, captionpos=b, label={lst:iperf}]
#---------------------------------------------------- 
#		Generacion de paquetes \textit{multicast}
# ---------------------------------------------------
# iperf -c 
# Flags:
#	-u utiliza UDP como protocolo de transporte
#	-T ttl para \textit{multicast}
#	-t tiempo durante el cual se tansmitira (segundos)
#	-i intervalo de tiempo entre reportes
iperf -c <IP Grupo> -u -T 32 -t 1000 -i 1

#---------------------------------------------------- 
#	  Suscripcion de un cliente SSM \textit{multicast}
# ---------------------------------------------------
# iperf -s
# Flags:
#	-u utiliza UDP como protocolo de transporte
#	-i intervalo de tiempo entre reportes
#	-B IP de grupo \textit{multicast}
#	-O Fuente para SSM \textit{multicast}
#	-X Interfaz para SSM \textit{multicast}
iperf -s -u -B <IP Grupo> -i 1 -O <IP Fuente> -X <Interfaz>
\end{lstlisting}







\subsection{Lighttpd}
A diferencia de otros servidores HTTP, como \textit{Apache} o \textit{Nginx}, \textit{Lighttpd} \parencite{lighttpd} es un servidor liviano en términos de consumo de memoria y carga del procesador, por eso se optó por esta opción.

\subsection{ONOS scripts}

En el cuadro \ref{table:onos-scripts}, se observan los \textit{scripts} que provee el proyecto de ONOS como herramientas para el desarrollador.

\begin{table}[th]
	\centering
	\begin{tabular}{| c | m{11cm} |}
		\hline		
		{\cellcolor[HTML]{C0C0C0}}\textbf{\textit{Nombre}} & \multicolumn{1}{c}{{\cellcolor[HTML]{C0C0C0}}\textbf{\textit{Descripción}}}                                                     \\
		\hline\hline		
		cell                     						   & Define las variables de entorno de \textit{bash} tales como, IP del controlador y usuario y contraseña para autenticarse con el controlador.                                                   \\
		\hline		
		{\cellcolor[HTML]{C0C0C0}}onos                     & {\cellcolor[HTML]{C0C0C0}}Permite conectarse a la línea de comando del controlador.                                                                                                                             \\
		\hline		
		onos-log                 						   & Permite visualizar los mensajes del controlador.                                                                                                                             \\
		\hline		
		{\cellcolor[HTML]{C0C0C0}}onos-secure-ssh          & {\cellcolor[HTML]{C0C0C0}}Permite establecer una conexión segura por ssh con la línea de comando. Remueve el usuario y contraseña por defecto y permite al usuario especificar su propio usuario y contraseña \parencite{onos-secure-ssh}. \\
		\hline		
		onos-app                                           & Permite instalar o desinstalar una aplicación en el controlador.                                                                                                                      \\
		\hline		
		{\cellcolor[HTML]{C0C0C0}}onos-create-app          & {\cellcolor[HTML]{C0C0C0}}Permite crear tu propio proyecto de una aplicación de ONOS. Incluyendo la posibilidad de agregar comandos a la CLI, entradas a la REST API y vistas a la GUI. \\                                         
		\hline		
	\end{tabular}
	\caption[Scripts de ONOS y sus funcionalidades]{Scripts de ONOS y sus funcionalidades \parencite{onosscripts}.}
	\label{table:onos-scripts}
\end{table}
